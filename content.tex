% DO NOT COMPILE THIS FILE DIRECTLY!
% This is included by the other .tex files.

% Page1
\begin{frame}[t,plain]
  \titlepage
\end{frame}

% Page2
\begin{frame}[t,fragile]{LaTeX Example - Bold text, comments and fragile}
\begin{itemize}
  \item This is {\bf bold text}
  \begin{lstlisting}
    {\bf bold text}
  \end{lstlisting}
but what is [t,fragile]?
fragile is needed for verbatim/lstlisting
  \item How to insert comments?
  Easy: \% this is a comment in LaTeX
  OR:
  \begin{lstlisting}
    \usepackage{verbatim}
    \begin{comment}
      \broken{LaTeX}
    \end{comment}
  \end{lstlisting}
  \end{itemize}
\end{frame}

% Page3
\begin{frame}[t,fragile]{LaTeX Example - New lines, center stuff, 3-dots}
\begin{itemize}
  \item How to insert new lines?
  Keep it real dude, this is not PowerPoint!
  \item How to center an image in portrait?
  \begin{lstlisting}
  \vfill
  \begin{center}
    \includegraphics[scale=0.2]{images/GotRum.jpg}
  \end{center}
  \vfill
  \end{lstlisting}
  \item Indentation rules: 2-spaces
  \item Three dots: $\dots$
  use Math mode:
  \begin{lstlisting}
  $\dots$
  \end{lstlisting}
  Or use this: \ldots
  \begin{lstlisting}
  \ldots
  \end{lstlisting}
\end{itemize}
\end{frame}

% Page4
\begin{frame}[t,fragile]{LaTeX Example - Tikz mind-map}
%\begin{tikzpicture}[thick,scale=0.6, every node/.style={scale=0.6}]
  \begin{center}
    \begin{tikzpicture}[thick,scale=0.5, every node/.style={transform shape}]
      \path[mindmap,concept color=black,text=white]
        node[concept] {Computer Science}
        [clockwise from=0]
        child[concept color=green!50!black] {
          node[concept] {practical}
          [clockwise from=90]
          child { node[concept] {algorithms} }
          child { node[concept] {data structures} }
          child { node[concept] {pro\-gramming languages} }
          child { node[concept] {software engineer\-ing} }
        }
        child[concept color=blue] {
          node[concept] {applied}
          [clockwise from=-30]
          child { node[concept] {databases} }
          child { node[concept] {WWW} }
        }
        child[concept color=red] { node[concept] {technical} }
        child[concept color=orange] { node[concept] {theoretical} };
    \end{tikzpicture}
  \end{center}
\end{frame}

% Page5
\begin{frame}[t,fragile]{LaTeX Example - URLs, umlaut}

  \begin{lstlisting}
     \url{}
  \end{lstlisting}

\url{http://wiki.localhost.lu}

Otherwise weird errors appear in your LaTeX console. You need to usepackage url

If you need some umlaut this nomenclature needs to be used:
  \begin{lstlisting}
     \"{e} OR $\ddot{e}$
  \end{lstlisting}

This renders to: \"{e} $\ddot{e}$ $\to$ use the $\backslash$"\{e\} notation, renders cleaner

More umlaut sweetness: $\grave{e}$, \`{e}, $\hat{e}$, \^{e}, $\acute{e}$, \'{e}

  \begin{lstlisting}
    $\grave{e}$ - \`{e}
    $\hat{e}$ - \^{e}
    $\acute{e}$ - \'{e}
  \end{lstlisting}

\end{frame}

% Page6
\begin{frame}[t,fragile]{LaTeX Example - simple math}

%So you want to write this: (a/x)\verb|^|2

The easy route would be:
\begin{lstlisting}
$(\frac{a}{x} )^2$
\end{lstlisting}

And you end up with this: $(\frac{a}{x} )^2$

But to beautify it we use \\left and \\right:

  \begin{lstlisting}
   $\left(\frac{a}{x} \right)^2$
  \end{lstlisting}
   $\left(\frac{a}{x} \right)^2$

\end{frame}

% Page7
\begin{frame}[t,fragile]{LaTeX Example - more math symbols}

%Some Math/symbols:

  \begin{lstlisting}
$\sqrt{x}$
$\times$
$\star$
$\le$
$\ge$
$\neq$
$\sim$
$\simeq$
$\approx$
  \end{lstlisting}

$\sqrt{x}$
$\times$
$\star$
$\le$
$\ge$
$\neq$
$\sim$
$\simeq$
$\approx$
\end{frame}

% Page8
\begin{frame}[t,fragile]{LaTeX Example - More symbols}
  \begin{lstlisting}
$\equiv$
$\infty$
$\to$
$\gets$
$\leftarrow$
$\rightarrow$
$\leftrightarrow$
$\Leftrightarrow$
$\backslash$
  \end{lstlisting}

$\equiv$
$\infty$
$\to$
$\gets$
$\leftarrow$
$\rightarrow$
$\leftrightarrow$
$\Leftrightarrow$
$\backslash$

\end{frame}

% Page9
\begin{frame}[t,fragile]{LaTeX Example - Greek letters}

%Greek Letters:
  \begin{lstlisting}
 \alpha \beta	 \gamma \delta \epsilon \varepsilon
 \zeta \eta	 \theta \vartheta \iota \kappa \lambda
 \mu	\nu \xi \pi \varpi \rho \varrho \sigma
 \varsigma \tau \upsilon \phi \varphi \chi \psi
 \omega
  \end{lstlisting}

\begin{tabular}{ l c c c c c c r }
  $\alpha$ & $\beta$ & $\gamma$ & $\delta$ & $\epsilon$ & $\varepsilon$ & $\zeta$ & $\eta$ \\
  $\theta$ & $\vartheta$ & $\iota$ & $\kappa$ & $\lambda$ & $\mu$ & $\nu$ & $\xi$ \\
  $\pi$ & $\varpi$ & $\rho$ & $\varrho$ & $\sigma$ & $\varsigma$ & $\tau$ & $\upsilon$ \\
  $\phi$ & $\varphi$ & $\chi$ & $\psi$ & $\omega$ \\
\end{tabular}

Uppercase letters start with an Uppercase letter :) $\alpha$

\end{frame}

% Page10
\begin{frame}[t,fragile]{LaTeX Example - Tables}

%Quick table:

\begin{lstlisting}
  \begin{tabular}{ l c c r }
  one & two & three & four \\
  5 & 6 & 7 & 8 \\
  9 & ten \\
  \end{tabular}
\end{lstlisting}

\begin{tabular}{ l | c || c r }
one & two & three & four \\ \hline
5 & 6 & 7 & 8 \\
9 & ten \\
\end{tabular}
\end{frame}

% Page 11
\begin{frame}[t,fragile]{LaTeX Example - quotes}

  `It\textquotesingle s a nice day!' \\
  OR \\
  `It{\textquotesingle}s a nice day!' \\

\end{frame}

% Page 12
\begin{frame}[t,fragile]{LaTeX Example - Upside down !}

  'To quickly turn a ? or ! upside down use the back-tick `! `? \\

\end{frame}

% Page 13
\begin{frame}[t,fragile]{LaTeX Example - Tikz scaling and drawing}

  \framebox{\begin{tikzpicture}[thick]
    \draw [dashed] (1,12) -- (11,12);
    \node[above] at (4,11) {1}; \node[above] at (8,11) {true};
  \end{tikzpicture}}

  \framebox{\begin{tikzpicture}[thick, scale=0.6]
    \draw [dashed] (1,12) -- (11,12);
    \node[above] at (4,11) {1}; \node[above] at (8,11) {true};
  \end{tikzpicture}}

  \framebox{\begin{tikzpicture}[thick, transform canvas={scale=0.6}]
    \draw [dashed] (1,12) -- (11,12);
    \node[above] at (4,11) {1}; \node[above] at (8,11) {true};
  \end{tikzpicture}}

  \framebox{\begin{tikzpicture}[thick,scale=0.6, every node/.style={scale=0.6}]
    \draw [dashed] (1,12) -- (11,12);
    \node[above] at (4,11) {1}; \node[above] at (8,11) {true};
  \end{tikzpicture}}

  \framebox{\begin{tikzpicture}[thick,scale=0.6, every node/.style={transform shape}]
    \draw [dashed] (1,12) -- (11,12);
    \node[above] at (4,11) {1}; \node[above] at (8,11) {true};
  \end{tikzpicture}}

\end{frame}

